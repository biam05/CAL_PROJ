%----------------------------------------------------------------------------------------
%	PACKAGES AND DOCUMENT CONFIGURATIONS
%----------------------------------------------------------------------------------------
\documentclass[article, a4paper, 12pt, oneside]{memoir}

% Margins
\usepackage[top=3cm,left=2cm,right=2cm,bottom=3cm]{geometry}

% Encondings
\usepackage[utf8]{inputenc}

% Language
\usepackage[portuguese]{babel}

% Graphics and images
\usepackage{graphicx}
	\graphicspath{{images/}}

% Tables
\usepackage{tabularx}

% Paragraph Spacing
\usepackage{parskip}
\usepackage{indentfirst}
\setlength{\parskip}{0.5cm}

% Hyperreferences
\usepackage{hyperref}

% Repeated commands
\usepackage{expl3}
\ExplSyntaxOn
\cs_new_eq:NN \Repeat \prg_replicate:nn
\ExplSyntaxOff

% Header and Footer Things
\usepackage{wallpaper}
\usepackage{fancyhdr}

% Following code to edit the pagestyle
\pagestyle{fancy}
\fancyhf{}
\rhead{WasteApp}
\lhead{\leftmark}
\rfoot{Página \thepage}

% Commands
\usepackage{xargs}

%% Linked Email
\newcommand{\email}[1]{
{\textup{\href{mailto:#1}{#1}} }
}

%----------------------------------------------------------------------------------------
%	DOCUMENT INFORMATION
%----------------------------------------------------------------------------------------
% Title
\title{\Huge \textup{WasteApp: recolha seletiva de lixo (Parte 1)} }
% Authors
\author{
\LARGE \textbf{Turma 1 - Grupo 1}\\\\
\begin{tabular}{l r}
	\email{up201806551@fe.up.pt} & Beatriz Costa Silva Mendes				\\
	\email{up201806524@fe.up.pt} & Daniel Garcia Lima Sarmento da Silva		\\
	\email{up201806538@fe.up.pt} & Henrique Manuel Ruivo Pereira			\\
	\Repeat{4}{\linebreak}
\end{tabular}
}

%\institute{Faculdade de Engenharia da Universidade do Porto \\ Concepção e Análise de Algoritmos - Turma 1, grupo 1}

% Date for the report
\date{24 de abril de 2020}

% Table of Contents
\addto\captionsportuguese{\renewcommand*\contentsname{Índice}}

%----------------------------------------------------------------------------------------
%	DOCUMENT
%----------------------------------------------------------------------------------------
\begin{document}
%----------------------------------------------------------------------------------------
%	Front Page
%----------------------------------------------------------------------------------------
% Title Author and Date
\maketitle

% More information for front page
\begin{center}
\textbf{Projeto CAL - 2019/20 - MIEIC}
\Repeat{2}{\linebreak}
\begin{tabular}{l r}
	\textbf{Professora das Aulas Laboratoriais}: & Liliana da Silva Ferreira
\end{tabular}
\Repeat{12}{\linebreak}
% FEUP Logo
\includegraphics[scale=0.3]{FEUPlogo.jpg}

\end{center}

\newpage
% Header Image
\CenterWallPaper{0.1}{FEUPlogo.jpg}
\addtolength{\wpXoffset}{-7.5cm}
\addtolength{\wpYoffset}{13.8cm}

\newpage

%----------------------------------------------------------------------------------------
%	TABLE OF CONTENTS
%----------------------------------------------------------------------------------------
\tableofcontents*

\newpage
%----------------------------------------------------------------------------------------
%									PRIMEIRA PARTE
%----------------------------------------------------------------------------------------
\thispagestyle{plain}

\begin{center}
\vspace*{\stretch{1}}
\LARGE Primeira Parte
\vspace*{\stretch{1}}
\end{center}

\newpage
%----------------------------------------------------------------------------------------
%	DESCRIÇÃO DO TEMA A SER IMPLEMENTADO
%----------------------------------------------------------------------------------------
\chapter[Descrição][Descrição]{Descrição} \label{\thechapter}

Neste trabalho pretende-se criar uma aplicação para gestão e localização dos pontos de recolha seletiva de resíduos, denominada \textit{WasteApp}. Esta \textit{app} deverá permitir aos seus utilizadores localizar os pontos de recolha seletiva mais próximos e os tipos de resíduos que lá se podem depositar, bem como realizar a gestão da sua capacidade.

Para além disso, a \textit{app} tem ainda por objetivo dar origem a um modelo de negócio que tem por base a recolha ao domicílio de determinados tipos de resíduos para exploração financeira (otimizando o itinerário percorrido).

A aplicação deve ser capaz de determinar a acessibilidade aos pontos de depósito/recolha.\textsuperscript{[1]}

\subsection[1ª Abordagem: Ponto de Recolha Mais Próximo][1ª Abordagem: Ponto de Recolha Mais Próximo]{1ª Abordagem: Ponto de Recolha Mais Próximo} \label{\thesection}
	
	Numa fase inicial, despreza-se a capacidade do ponto de recolha e os tipos de resíduos que poderão ser depositados neste. Deste modo, o único objetivo acaba por ser apenas determinar qual o ponto de recolha mais próximo do utilizador, calculando a rota mais curta até um qualquer ponto de recolha (partindo-se do princípio de que este se encontra acessível).

\section[2ª Abordagem: Ponto de Recolha de um Determinado Resíduo Mais Próximo][2ª Abordagem: Ponto de Recolha de um Determinado Resíduo Mais Próximo]{2ª Abordagem: Ponto de Recolha de um Determinado Resíduo Mais Próximo} \label{\thesection}
	
	Neste segundo passo, acrescenta-se ao problema o facto de que determinados pontos de recolha se limitam a aceitar certos tipos de resíduos. Assim, cada utilizador terá de ter pelo menos 5 pontos de recolha mais próximos, um para cada tipo de resíduo (papel, vidro, plástico, pilhas e lixo indiferenciado).	


\section[3ª Abordagem: Ponto de Recolha de um Determinado Resíduo Mais Próximo com Capacidade Suficiente][3ª Abordagem: Ponto de Recolha de um Determinado Resíduo Mais Próximo com Capacidade Suficiente]{3ª Abordagem: Ponto de Recolha de um Determinado Resíduo Mais Próximo com Capacidade Suficiente} \label{\thesection}

	
	Posteriormente, é necessário considerar que os pontos de recolha têm uma capacidade máxima, ou seja, depois de atingir essa capacidade, não poderá ser depositado nele mais nenhum resíduo. Assim sendo, se o ponto de recolha mais próximo do utilizador de um determinado resíduo ultrapassar a sua capacidade máxima com o depósito do utilizador, terá de ser atribuído a este um novo ponto de recolha desse mesmo resíduo, que será o mais próximo ainda com capacidade.

\section[4ª Abordagem: Implementação Paralela do Modelo de Negócio][4ª Abordagem: Implementação Paralela do Modelo de Negócio]{4ª Abordagem: Implementação Paralela do Modelo de Negócio} \label{\thesection}
	
	Por fim, nesta última abordagem, terá de ser implementado um serviço de recolha de resíduos ao domicílio para exploração financeira, que depois serão levados para uma central de reciclagem. Assim sendo, terá de ser determinado o menor itinerário possível que passe pelas casas dos utilizadores que fornecem os resíduos a ser recolhidos.



\newpage
%----------------------------------------------------------------------------------------
%	IDENTIFICAÇÃO E FORMALIZAÇÃO DO PROBLEMA
%----------------------------------------------------------------------------------------
\chapter[Identificação e formalização do problema][Identificação e formalização do problema]{Identificação e formalização do problema} \label{\thechapter}

\section{Ponto de Recolha Mais Próximo}

\subsection{Identificação do Problema}

Nesta vertente de utilização da \textit{app}, pretende-se disponibilizar ao utilizador o ponto de recolha mais próximo em que pode depositar o tipo e a quantidade de lixo que pretende, bem como o itinerário para lá chegar.

\subsection{Dados de Entrada}

\begin{itemize} %1

\item $type$ – tipo de lixo que se pretende depositar (plástico, vidro, papel, etc.);
\item $quantity$ – quantidade de lixo que se pretende depositar;
\item $G = (V, E)$ – grafo dirigido pesado, composto por:

	\begin{itemize} %2

	\item $V$ – vértices, que representam interseções, compostos por:

		\begin{itemize} %3
		\item $Dist$ – distância a L;
		\item $Adj \subseteq E$ – conjunto de arestas que partem do vértice.
		\end{itemize} %3

	\end{itemize} %2

	\begin{itemize} %2

	\item $E$ – arestas, que representam vias de comunicação, compostas por:

		\begin{itemize} %3
		\item $W$ – peso da aresta (distância entre os dois vértices);
		\item $ID$ – identificador único da aresta;
		\item $V\textsubscript{i} \in V$ – vértice de partida;
		\item $V\textsubscript{f} \in V$ – vértice de chegada.
		\end{itemize} %3

	\end{itemize} %2

\item $P$ – conjunto de pontos de recolha, cada um caracterizado por:

	\begin{itemize} %2

	\item $type$ – tipo de lixo que recolhem;
	\item $Q$ – quantidade do lixo que lá está;
	\item $Q \textsubscript{max}$ - quantidade máxima do lixo que pode lá estar;
	\item $R \in E$ – aresta onde está;
	\item $D$ – distância ao vértice de chegada da aresta.

	\end{itemize} %2

\item $L$ - Localização do utilizador, caracterizada por:
	
	\begin{itemize}
	
	\item $R \in E$ - aresta onde está;
	\item $D$ - distância ao vértice de chegada da aresta.
	
	\end{itemize}

\end{itemize} %1

\subsection{Dados de Saída}

\begin{itemize}
\item $P\textsubscript{f} \in P$ - ponto de recolha do tipo correspondente mais próximo do utilizador com capacidade para a quantidade de lixo que se pretende depositar;
\item $C = {e \in E: 1 \leq i \leq |E|}$ - sequência de arestas, correspondente ao caminho a percorrer (e\textsubscript{i} = i-ésima aresta).
\end{itemize}


\subsection{Restrições}

\begin{itemize}

\item Dados de Entrada:

	\begin{itemize}
	\item $quantity > 0$, visto que representa peso ou volume;
	\item $\forall  p \in P, Q \geq 0 \land Q\textsubscript{max} > 0 \land Q \leq Q\textsubscript{max}$, pelo mesmo motivo;
	\item $\forall  e \in E, W > 0$, visto que o peso das arestas representa uma distância;
	\item $\forall L \land \forall p \in P, D \geq 0$, pelo mesmo motivo;
	\item $\forall v \in V, Dist \geq 0$, pelo mesmo motivo.
	\end{itemize}
	
\item Dados de Saída:

	\begin{itemize}
	\item $R(P\textsubscript{f}) = e\textsubscript{final}$.
	\end{itemize}
	
\end{itemize}


\subsection{Funções Objetivo}

A solução ótima do problema passa por minimizar a distância que o utilizador tem de percorrer para depositar o lixo, de modo que pretende-se minimizar a seguinte função:

$$f = \sum_{c \in C} W(c)$$\\


\section{Novo Modelo de Negócio}

\subsection{Identificação do Problema}

Nesta vertente da \textit{app}, pretende-se disponibilizar ao utilizador um itinerário desde a sua localização até à localização da central de reciclagem, passando por domicílios que tenham o tipo de resíduos que pretendem recolher, sem nunca ultrapassar a quantidade possibilitada pelo utilizador.

\subsection{Dados de Entrada}

\begin{itemize} %1

\item $type$ – tipo de lixo que se pretende depositar (plástico, vidro, papel, etc.);
\item $quantity$ – quantidade de lixo que se pretende recolher;
\item $G = (V, E)$ – grafo dirigido pesado, composto por:

	\begin{itemize} %2

	\item $V$ – vértices, que representam interseções, compostos por:

		\begin{itemize} %3
		\item $Dist$ – distância a L\textsubscript{i};
		\item $Adj \subseteq E$ – conjunto de arestas que partem do vértice.
		\end{itemize} %3

	\end{itemize} %2

	\begin{itemize} %2

	\item $E$ – arestas, que representam vias de comunicação, compostas por:

		\begin{itemize} %3
		\item $W$ – peso da aresta (distância entre os dois vértices);
		\item $ID$ – identificador único da aresta;
		\item $V\textsubscript{i} \in V$ – vértice de partida;
		\item $V\textsubscript{f} \in V$ – vértice de chegada.
		\end{itemize} %3

	\end{itemize} %2

\item $P\textsubscript{i}$ – conjunto de domicílios que pretendem recolha, cada um caracterizado por:

	\begin{itemize} %2

	\item $type$ – tipo de lixo que recolhem;
	\item $Q$ – quantidade do lixo que lá está;
	\item $R \in E$ – aresta onde está;
	\item $D$ – distância ao vértice de chegada da aresta.

	\end{itemize} %2

\item $L\textsubscript{i}$ – localização do utilizador, caracterizada por:

	\begin{itemize} %2
	\item $R \in E$ – aresta onde está;
	\item $D$ – distância ao vértice de chegada da aresta.
	\end{itemize} %2
	
\item $L\textsubscript{f}$ – localização da central de reciclagem:

	\begin{itemize} %2
	\item $R \in E$ – aresta onde está;
	\item $D$ – distância ao vértice de chegada da aresta.
	\end{itemize} %2
	
\end{itemize} %1

\subsection{Dados de Saída}

\begin{itemize}
\item $C = {e \in E: 1 \leq i \leq |E|}$ - sequência de arestas, correspondente ao caminho a percorrer (e\textsubscript{i} = i-ésima aresta);
\item $P\textsubscript{f} = {p \in P: 1 \leq i \leq |P|}$ - sequência de domicílios de onde o utilizador deve recolher resíduos.
\end{itemize}

\subsection{Restrições}

\begin{itemize}

\item Dados de Entrada:

	\begin{itemize}
	\item $quantity > 0$, visto que representa peso ou volume;
	\item $\forall  p \in P, Q \geq 0$, pelo mesmo motivo;
	\item $\forall  e \in E, W > 0$, visto que o peso das arestas representa uma distância;
	\item $\forall L \land \forall p \in P, D \geq 0$, pelo mesmo motivo;
	\item $\forall v \in V, Dist \geq 0$, pelo mesmo motivo.
	\end{itemize}
	
\item Dados de Saída:

	\begin{itemize}
	\item $\forall p \in P\textsubscript{f}, R(p) \in C$.
	\item $\sum_{p \in P\textsubscript{f}} Q(p) < quantity$.
	\end{itemize}

\end{itemize}

\subsection{Funções Objetivo}

A solução do problema passa por maximizar a quantidade recolhida, sem ultrapassar a quantidade possível de recolher, e por minimizar a distância percorrida, dando prevalência ao primeiro critério. Assim, queremos minimizar as seguintes funções:

$$f = quantity - \sum_{p \in P\textsubscript{f}} Q(p)$$, sendo que nunca pode ser $<$ 0

$$g = \sum_{c \in C} W(c)$$

\newpage
%----------------------------------------------------------------------------------------
%	PERSPETIVA DE SOLUÇÃO
%----------------------------------------------------------------------------------------
\chapter[Perspetiva de Solução][Perspetiva de Solução]{Perspetiva de Solução} \label{\thechapter}
A perspetiva de solução adotada tem por base a aplicação de várias fases prévias ao processamento do problema.

Para reduzir a complexidade temporal do processamento, os passos iniciais baseiam-se na eliminação de arestas que não proporcionem nenhum tipo de vantagem (tais como as correspondentes a vias inutilizáveis - por obras, por exemplo - e as que tenham arestas/conjuntos de arestas com origem e destino comuns de peso total menor).

O passo seguinte corresponde à ordenação dos pontos de recolha por proximidade a cada utilizador. Tendo em conta que a morada de um utilizador e os pontos de recolha podem corresponder a um nó ou a uma parte de uma aresta, se se tratar do primeiro caso apenas se tem de ter em conta as arestas adjacentes; quanto ao segundo caso, terá de se localizar o ponto de recolha ou a morada dentro da aresta, isto é, saber a distância deste(a) a cada uma das extremidades da aresta.\\\

\section{Ponto de Recolha Mais Próximo}

Para o primeiro caso, usa-se o algoritmo de \textit{Dijsktra} abordado nas aulas, cujo pseudocódigo é apresentado na figura seguinte, uma vez que o grafo não contem arestas de peso negativo. \\\

\begin{figure}[h!]
  \centerline{\includegraphics[scale=1]{Dijkstra_pseudocode.jpg}}
  \caption{Pseudocódigo do Algoritmo de \textit{Dijsktra}.}
\end{figure}

A utilização deste algoritmo de cálculo do caminho mais curto desde o vértice inicial (morada do utilizador) até aos outros nós (pontos de recolha) resulta numa árvore de caminhos ordenada.

A preparação do algoritmo tem complexidade temporal O($|V|$) e consiste na inicialização do campo \textbf{dist} a $\infty$ e \textbf{path} a "nil" (correspondente a \textit{nullptr}) em todos os vértices (à exceção da origem).

Posteriormente, a partir do vértice inicial, percorre-se todas as arestas adjacentes e adicionam-se os vértices de destino a uma fila de prioridade, atualizando a dist (distância do vértice anterior somada com o peso da aresta) e o path (vértice anterior), se este vértice ainda não tiver sido visitado, ou se a distância obtida for menor do que o valor prévio de dist.

Este processo tem complexidade temporal O($(|V| + |E|)*log(|V|)$), uma vez que a cada passo - O($|V| + |E|$) - pode ser necessário adicionar, remover ou mover elementos na fila de prioridade - O($log(|V|)$).
	
Para o segundo caso, ter-se-ia de aplicar uma "translação" ao mesmo algoritmo (uma vez que não se teria de calcular a distância de um nó a todos os outros, mas de um potencial ponto interno da aresta até um potencial ponto interno de outra aresta), tendo que, para todas as distâncias calculadas a partir de cada uma das extremidades dessa aresta (pelo algoritmo referido), somar-se a distância do ponto interno a essa mesma extremidade.

Posteriormente, para implementar a especificidade dos resíduos, seria apenas necessário, em cada ponto de recolha, guardar o(s) tipo(s) de resíduos nele recolhidos.

Após essa implementação, resta ter em consideração o fator capacidade, a ser guardado também em cada ponto de recolha e para cada um dos resíduos recolhidos.

\section{Novo Modelo de Negócio}

O processamento do novo modelo de negócio proposto baseia-se no cálculo do itinerário mais curto até uma central de reciclagem, passando por todas as casas que desejam ver esse tipo de resíduo recolhido.

Este problema assemelha-se ao \textbf{\textit{Travelling Salesman Problem} (TSP)}\textsuperscript{[2]}, pelo que os métodos a aplicar serão semelhantes às resoluções propostas para este. As três opções ponderadas consistem em:

\begin{itemize}
\item \textbf{Força bruta:} Esta alternativa consiste no cálculo de todas as opções possíveis e seleção da mais rentável. Obviamente, este é o método mais temporalmente complexo (O($|V|!$)) e, portanto, apesar de garantir um resultado ótimo, é pouco recomendável.

\item \textbf{Algoritmo do Vizinho mais Próximo:} Este algoritmo baseia-se na ordenação dos pontos, encontrando o ponto não visitado mais próximo do anterior processado. Este algoritmo não garante um resultado ótimo, mas tem complexidade bastante inferior ao anterior (complexidade temporal O($|V|^2$) no pior caso).\\

\item \textbf{Algoritmo de \textit{Held-Karp}:} Este é um algoritmo de programação dinâmica baseado na recursividade da distância mínima, no qual para cada vértice, se calcula a distância do vértice inicial até esse passando por todos os vértices de um subconjunto. A sua complexidade temporal é de O($2^n n^2$) e o algoritmo garante um resultado ótimo. O pseudocódigo do algoritmo de \textit{Held-Karp} é apresentado na figura seguinte.\textsuperscript{[3]}
\end{itemize}

\begin{figure}[h!]
  \centerline{\includegraphics[scale=1]{Held-Karp Algorithm.PNG}}
  \caption{Pseudocódigo do Algoritmo de \textit{Held-Karp}.}
\end{figure}

\newpage

%----------------------------------------------------------------------------------------
%	CASOS DE UTILIZAÇÃO
%----------------------------------------------------------------------------------------
\chapter[Casos de Utilização][Casos de Utilização]{Casos de Utilização} \label{\thechapter}

A aplicação \textit{Wasteapp} que vamos implementar terá como base uma interface simples de texto com a qual o utilizador poderá interagir. Deste modo, para facilitar a interação, esta terá um conjunto de menus com várias opções a serem disponibilizadas, dividindo-se em 2 menus gerais: 1 menu para o utilizador comum que procura os pontos de recolha mais próximos e 1 menu para o utilizador que irá explorar financeiramente o novo modelo de negócio.
Para além disso, será possível guardar a informação de um mapa sob a forma de um grafo através da leitura de ficheiros de texto, no qual incidirão as diferentes funcionalidades da \textit{app}, dentro das quais se destacam:

\begin{itemize}
\item Visualização dos dados do grafo e da informação representada por este (vias de trânsito, pontos de recolha, etc.) no \textit{GraphViewer};
\item Ordenação dos pontos de recolha por proximidade ao utilizador e do itinerário ótimo para o modelo de negócio;
\item Consulta da informação sobre os pontos de recolha disponíveis (capacidade e tipo de resíduo);
\item Verificação da acessibilidade dos pontos de recolha (verificação se as ruas se encontram desimpedidas, sem estarem em obras, por exemplo);
\item Ordenação e atribuição das casas dos utilizadores que solicitaram a recolha de resíduos ao domicílio.

\end{itemize}

\begin{figure}[h!]
  \centerline{\includegraphics[scale=0.6]{graphviewer.png}}
  \caption{Exemplo de um grafo visualizado no \textit{GraphViewer}.}
\end{figure}


\newpage
%----------------------------------------------------------------------------------------
%	CONCLUSÃO
%----------------------------------------------------------------------------------------
\chapter[Conclusão (Primeira Parte)][Conclusão (Primeira Parte)]{Conclusão (Primeira Parte)} \label{\thechapter}

Após um estudo rigoroso do problema que nos foi atribuído, percebemos que existem diversas maneiras para resolver problemas deste género. Depois do aconselhamento por parte da professora das aulas práticas e dos monitores, acreditamos que obtivemos uma abordagem correta para a solução do problema de identificação dos itinerários ótimos.

A nosso ver, a parte que requereu mais tempo foi o ponto "Perspetiva de Solução" uma vez que necessitou de bastante pesquisa em diversas plataformas de forma a encontrarmos o algoritmo mais apropriado para a solução dos problemas que surgiram, pois serão utilizados diferentes algoritmos tanto na utilização "normal" da aplicação como na perspetiva de novo modelo de negócio que surge com esta.

Por fim, podemos concluir que, após a realização deste relatório, dominamos agora a matéria lecionada em aula que incide na manipulação de grafos com algoritmos, encontrando-nos, assim, preparados para a implementação do que sugerimos na parte do trabalho que se segue.


\newpage
%----------------------------------------------------------------------------------------
%	CONTRIBUIÇÃO
%----------------------------------------------------------------------------------------
\chapter[Contribuição (Primeira Parte)][Contribuição (Primeira Parte)]{Contribuição (Primeira Parte)} \label{\thechapter}

\textbf{Beatriz Costa Silva Mendes} - up201806551@fe.up.pt

\begin{itemize}
\item Contribuição: $1/3$
\item Tarefas:
	\begin{itemize}
	\item Descrição do Tema;
	\item Casos de Utilização;
	\item Colaboração na Perspetiva de Solução.
	\end{itemize}
\end{itemize}


\textbf{Daniel Garcia Lima Sarmento da Silva} - up201806524@fe.up.pt

\begin{itemize}
\item Contribuição: $1/3$
\item Tarefas:
	\begin{itemize}
	\item Formalização do Problema;
	\item Conclusão.
	\end{itemize}
\end{itemize}


\textbf{Henrique Manuel Ruivo Pereira} - up201806538@fe.up.pt

\begin{itemize}
\item Contribuição: $1/3$
\item Tarefas:
	\begin{itemize}
	\item Colaboração na Descrição do Tema;
	\item Colaboração nos Casos de Utilização;
	\item Perspetiva de Solução
	\end{itemize}
\end{itemize}

\newpage
%----------------------------------------------------------------------------------------
%									SEGUNDA PARTE
%----------------------------------------------------------------------------------------
\thispagestyle{plain}

\begin{center}
\vspace*{\stretch{1}}
\LARGE Segunda Parte
\vspace*{\stretch{1}}
\end{center}

\newpage
%----------------------------------------------------------------------------------------
%	PRINCIPAIS CASOS DE USO IMPLEMENTADOS
%----------------------------------------------------------------------------------------
\chapter[Principais Casos de Uso Implementados][Principais Casos de Uso Implementados]{Principais Casos de Uso Implementados} \label{\thechapter}

A aplicação \textit{WasteApp} desenvolvida pelo nosso grupo de trabalho tem como base uma interface com a qual o utilizador poderá interagir. 

Inicialmente, o utilizador terá de selecionar que mapa pretende ler e os ficheiros que terão de ser analisados. Existem duas opções: mapa \textbf{4x4} (utilizado para testar os algoritmos implementados) e o mapa \textbf{Porto}. Cada um destes está guardado dentro da pasta \textit{data} e tem um ficheiros de nós, \textit{nodes.txt}, de arestas \textit{edges.txt} e de utilizadores da aplicação, \textit{userlogins.txt}. O ficheiro dos pedidos de cada cliente está guardado na pasta \textit{data} e é comum aos dois mapas. 

\begin{figure}[h!]
  \centerline{\includegraphics[scale=0.8]{wasteapp_1stmenu.png}}
  \caption{Primeiro Menu da aplicação \textit{WasteApp}.}
\end{figure}

Posteriormente, o utilizador terá selecionar se pretende fazer \textit{login} na sua conta ou criar uma conta nova. Estas opções são necessárias uma vez que o utilizador poderá ser um utilizador "normal" que tem como objetivo saber qual o ponto de recolha mais próximo de sua casa ou um utilizador que pretende ir a casa dos restantes recolher os resíduos, logo terão de ser apresentados menus diferentes com funcionalidades igualmente diferentes.

\begin{figure}[h!]
  \centerline{\includegraphics[scale=0.8]{wasteapp_2ndmenu.png}}
  \caption{Segundo Menu: \textit{Login} ou Sign In.}
\end{figure}

Ao selecionar a opção de \textbf{\textit{login}}, o utilizador terá de escrever o seu \textit{username} e a sua \textit{password}. Poderão acontecer três situações distintas:

\begin{itemize}
	\item Se for encontrado no ficheiro de \textit{users} o utilizador com esse \textit{username} e \textit{password}, ele será redirecionado para a sua página (ou de cliente ou de trabalhador);
	\item Se for encontrado um utilizador com esse \textit{username} mas a \textit{password} não for igual à que está guardada, terá a oportunidade de digitar novamente os seus dados;
	\item Se não for encontrado um utilizador com o \textit{username} dado, este poderá tentar outra vez colocar os dados ou poderá criar uma conta nova (será redirecionado para a página de \textit{sign in}).
\end{itemize}

\begin{figure}[h!]
  \centerline{\includegraphics[scale=0.8]{wasteapp_login.png}}
  \caption{Exemplo de um \textit{Login}.}
\end{figure}

Por outro lado, se o utilizador selecionar a opção de \textit{\textbf{sign in}}, este terá de dar algumas das suas credenciais: o \textbf{\textit{username}} que pretende utilizar na aplicação, uma \textbf{\textit{password}}, um \textbf{vértice} que será a sua morada e se pretende ser um \textbf{trabalhador ou cliente}

\begin{figure}[h!]
  \centerline{\includegraphics[scale=0.8]{wasteapp_signin.png}}
  \caption{Exemplo de um \textit{Sign In}.}
\end{figure}

A informação relativamente aos pontos de interesse é guardada num grafo, sendo que estes são os vértices (nem todos os vértices são pontos de interesse, no entanto todos os pontos de interesse são vértices). Este grafo é gerado e apresentado no ecrã do utilizador com o auxílio da API \textit{GraphViewer} dependendo das opções que este seleciona ao longo do programa. Poderá ter duas visualizações distintas:

\begin{itemize}
	\item Se o utilizador for um \textbf{cliente} e quiser saber qual o ponto de recolha de um determinado resíduo mais perto da sua habitação, este terá de digitar o resíduo que pretende depositar e a quantidade deste, uma vez que os pontos de recolha poderão já ter atingido a sua capacidade máxima. Após a introdução dos dados, é apresentado \textit{GraphViewer}:
	\begin{itemize}
		\item o \textbf{grafo} lido;
		\item um \textbf{ponto verde} - casa do utilizador;
		\item um \textbf{ponto azul} - o ponto de recolha do resíduo escolhido ainda com capacidade mais próximo da habitação;
		\item \textbf{caminho vermelho} - o caminho que o utilizador teria de tomar para chegar ao ponto determinado. 
	\end{itemize}
	
	\begin{figure}[h!]
  		\centerline{\includegraphics[scale=0.4]{wasteapp_4x4henrique123organic.png}}
  		\caption{Grafo gerado no \textit{GraphViewer} quando o utilizador escolho o mapa 4x4 e dá como \textit{input} "Organic" e "3".}
	\end{figure}
	
	\item Se o utilizador for um \textbf{trabalhador} e quiser saber qual o percurso que terá de tomar para recolher determinados resíduos às casas dos utilizadores que realizaram o pedido, este terá de digitar o resíduo que pretende depositar. Após a introdução dos dados, é apresentado \textit{GraphViewer}:
	\begin{itemize}
		\item o \textbf{grafo} lido;
		\item um \textbf{ponto verde} - casa do trabalhador;
		\item um \textbf{ponto azul} - central de recolha que torna o percurso gerado mais curto;
		\item \textbf{caminho vermelho} - o caminho que o utilizador teria de percorrer de forma a passar em todas as casas possíveis onde os utilizadores fizeram pedidos de recolha, sendo que este será o caminho mais curto possível;
		\item \textbf{pontos amarelos} - casas das pessoas que pretendem que os resíduos sejam recolhidos à sua porta.
		
		{\Large FALTA IMAGEM COM PONTOS AMARELOS!!!}
		
	\end{itemize}
	
\end{itemize}


\newpage
%----------------------------------------------------------------------------------------
%	ESTRUTURAS DE DADOS UTILIZADAS E CLASSES UTILIZADAS
%----------------------------------------------------------------------------------------
\chapter[Estruturas de Dados  e Classes Utilizadas][Estruturas de Dados e Classes Utilizadas]{Estruturas de Dados e Classes Utilizadas} \label{\thechapter}

\section{Representação do Grafo}

\subsection{Grafo}

A representação do grafo foi feita com base na classe \textbf{\textit{WasteApp}}, definida em \textit{wasteapp.h}.  Esta estrutura é constituída por um \textbf{vetor de utilizadores}, um \textbf{vetor de vértices} e um \textbf{vetor de arestas}, sendo os dois últimos os utilizados na representação do grafo. O vetor é necessário para se poder realizar uma iteração sobre todo o conjunto de vértices do grafo. Possui diversos métodos para a manipulação deste, algoritmos de pesquisa e ainda um método que possibilita a visualização do grafo.

\subsection{Aresta}

As arestas do grafo são representadas pela classe \textbf{\textit{Edge}}, definida em \textit{edge.h}. Cada aresta é constituída por 4 parâmetros: um \textbf{peso}, um \textbf{ID}, o \textbf{vértice inicial} e o \textbf{vértice de chegada}).

\subsection{Vértice}

Os vértices utilizados na representação do grafo são representados pela classe \textbf{\textit{Vertex}}, definida em \textit{vertex.h}. Esta estrutura de dados possui uma \textit{distância} ao utilizador (utilizado para determinar o percurso mais curto), um \textbf{vetor} com todos os IDs das arestas que estão ligadas a este vértice, uma \textbf{coordenada horizontal}, uma \textbf{coordenada vertical} e um \textbf{ID} que o identifica.

\section{Desenho dos Mapas e Itinerários}

A classe \textbf{\textit{WasteApp}} é responsável também por, utilizando o \textit{GraphViewer}, representar no ecrã os mapas, itinerários e grafos resultantes dos algoritmos utilizados ao longo da aplicação.

\section{WasteApp}

\subsection{Serviços}

A classe \textbf{\textit{WasteApp}} é também responsável pela gestão de serviços da aplicação. Deste modo, esta classe tem como atributos os 3 vetores mencionados anteriormente.

\subsection{Utilizadores}

Os utilizadores da aplicação poderão ser de dois tipos: \textit{\textbf{CLIENTE}} ou \textbf{\textit{WORKER}}, estando estes definidos numa enumeração de nome \textit{userType}. Esta classe é constituída pelo \textit{\textbf{username}} que o utilizador tem na aplicação, a sua \textit{\textbf{password}}, uma \textbf{casa}, que será a posição a ter em conta caso este queira ir para o local de recolha mais próximo ou para requisitar a função de recolha de lixo ao domicílio da aplicação, e o \textbf{tipo} de utilizador que é.

\subsection{Lixo a ser Depositado pelo Utilizador}

Quando um cliente pretende saber em que local poderá depositar os seus resíduos, é gerado um pedido q é guardado na classe \textbf{\textit{Request}}, definida em \textit{request.h}. Esta estrutura de dados é constituída por um \textbf{tipo de lixo} que irá ser depositado, uma \textbf{quantidade} e o \textbf{utilizador} que pretende fazê-lo, de modo a ser possível calcular o caminho mais curto deste até a um ponto de recolha

\subsection{Casa}

A classe \textbf{\textit{House}}, definida em \textit{house.h}, é constituída por um único parâmetro que é o \textbf{vértice} onde esta se encontra.

\subsection{Pedido de Recolha de Resíduos ao Domicílio}

Os pedidos de recolha de resíduos ao domicílio são representados pela classe \textbf{\textit{HouseRequest}}, definida em \textit{houserequest.h}. Esta classe é constituída por um \textbf{tipo de lixo} que vai se recolhido e uma \textbf{quantidade} de lixo a ser recolhido.

\subsection{Ponto de Recolha}

A classe \textbf{\textit{Spot}}, definida em \textit{spot.h}, é constituída por quatro parâmetros: um \textbf{tipo} de resíduos que poderá recolher, a \textit{quantidade} que terá neste (iniciada a 0 e atualizada ao longo do programa), a \textit{capacidade} total deste (20 em todos os spots) e o \textbf{vértice} em que se encontra.

\newpage
%----------------------------------------------------------------------------------------
%	ALGORITMOS EFETIVAMENTE IMPLEMENTADOS
%----------------------------------------------------------------------------------------
\chapter[Algoritmos Efetivamente Implementados][Algoritmos Efetivamente Implementados]{Algoritmos Efetivamente Implementados} \label{\thechapter}





\section{Análise de complexidade dos algoritmos efetivamente implementados}

\subsection{Análise teórica (temporal e espacial)}

\subsection{Análise temporal empírica }

\newpage
%----------------------------------------------------------------------------------------
%	CONECTIVIDADE DOS GRAFOS UTILIZADOS
%----------------------------------------------------------------------------------------
\chapter[Conectividade dos Grafos Utilizados][Conectividade dos Grafos Utilizados]{Conectividade dos Grafos Utilizados} \label{\thechapter}

\textsuperscript{[4]} \textsuperscript{[5]}







\newpage
%----------------------------------------------------------------------------------------
%	CONCLUSÃO GERAL
%----------------------------------------------------------------------------------------
\chapter[Conclusão Geral][Conclusão Geral]{Conclusão Geral} \label{\thechapter}







\newpage
%----------------------------------------------------------------------------------------
%	CONTRIBUIÇÃO (SEGUNDA PARTE)
%----------------------------------------------------------------------------------------
\chapter[Contribuição (Segunda Parte)][Contribuição (Segunda Parte)]{Contribuição (Segunda Parte)} \label{\thechapter}

\textbf{Beatriz Costa Silva Mendes} - up201806551@fe.up.pt

\begin{itemize}
\item Contribuição: $1/3$
\item Tarefas:
	\begin{itemize}
	\item Interface com o Utilizador
	\item Criação de Pontos de Interesse
	\item Criação de Classes fundamentais no desenvolvimento do projeto
	\item Atualizações no Relatório	
	\end{itemize}
\end{itemize}


\textbf{Daniel Garcia Lima Sarmento da Silva} - up201806524@fe.up.pt

\begin{itemize}
\item Contribuição: $1/3$
\item Tarefas:
	\begin{itemize}
	\item Implementação da interface do \textit{GraphViewer}
	\item Criação de Classes fundamentais no desenvolvimento do projeto
	\item Atualizações no Relatório
	\end{itemize}
\end{itemize}


\textbf{Henrique Manuel Ruivo Pereira} - up201806538@fe.up.pt

\begin{itemize}
\item Contribuição: $1/3$
\item Tarefas:
	\begin{itemize}
	\item Interface com o Utilizador
	\item Algoritmos
	\item Criação de Classes fundamentais no desenvolvimento do projeto
	\item Atualizações no Relatório
	\end{itemize}
\end{itemize}


\newpage
%----------------------------------------------------------------------------------------
%	REFERÊNCIAS BIBLIOGRÁFICAS
%----------------------------------------------------------------------------------------
\chapter[Referências Bibliográficas][Referências Bibliográficas]{Referências Bibliográficas} \label{\thechapter}
Grupo B, Turma 6, 2016. \textit{Ecoponto: Recolha De Lixo Seletiva (Tema 4) ‐ Parte 1}. Projeto de Concepção e Análise de Algoritmos.

[1]: Docs.google.com. 2020. 2019-20 2S CAL Trabalhos Publicados. Disponível no Moodle.

[2]: Wikipedia Contributors (2019). \textit{Travelling salesman problem}. [online] Wikipedia. Disponível em: \url{https://en.wikipedia.org/wiki/Travelling_salesman_problem}.

[3]: Wikipedia Contributors (2019). \textit{Held–Karp algorithm}. [online] Wikipedia. Disponível em: \url{https://en.wikipedia.org/wiki/Held%E2%80%93Karp_algorithm}.

[4] Rossetti, R., Ferreira, L., Cardoso, H. L. e Andrade, F., 2020. \textit{Algoritmos Em Grafos: Conectividade}.

[5] \textit{Data Structures and Algorithm Analysis in Java}, Second
Edition, Mark Allen Weiss, Addison Wesley, 2006 

Figura 1: Rossetti, R., Ferreira, L., Cardoso, H. L. e Andrade, F., 2020. \textit{Algoritmos Em Grafos: Caminho Mais Curto (Parte I)}.

Figura 2: Wikipedia Contributors (2019). \textit{Held–Karp algorithm}. [online] Wikipedia. Disponível em: \url{https://en.wikipedia.org/wiki/Held%E2%80%93Karp_algorithm}.

Figura 3: GitHub. (2020). \textit{STEMS-group/GraphViewer}. [online] Disponível em: \url{https://github.com/STEMS-group/GraphViewer}.

‌

‌

\newpage
\end{document}
